\documentclass[11pt]{article}
%\pdfoutput=1
% ------------------------------------------------------------ %

\usepackage{amsmath}

\usepackage{authblk} % for affiliations
\renewcommand\Affilfont{\itshape\footnotesize}
\setlength{\affilsep}{1em}

\usepackage[margin=10pt, font=small, labelfont=bf, tableposition=top]{caption}  % adjust caption properties

\usepackage[margin=1in]{geometry} % document margin
\usepackage{graphbox}
\usepackage{graphicx} 

\usepackage{hyperref}
\hypersetup{colorlinks=true, breaklinks=true, linkcolor=darkblue, menucolor=darkblue, urlcolor=darkblue, citecolor=darkblue}

%\usepackage{libertine}

\usepackage{mathtools} 
\DeclarePairedDelimiter{\ceil}{\lceil}{\rceil}

% enhanced tables
\usepackage{multicol}
\usepackage{multirow}
\usepackage{booktabs}
\usepackage{tabularx}
\newcolumntype{L}[1]{>{\raggedright\arraybackslash}p{#1}} %
\newcolumntype{C}[1]{>{\centering\arraybackslash}p{#1}} % 
\newcolumntype{R}[1]{>{\raggedleft\arraybackslash}p{#1}} % 

\usepackage{natbib} % bibliography
\bibpunct{(}{)}{,}{a}{}{,}  % adjust punctuation in references



\usepackage{tikz}

\usepackage{xcolor}
\definecolor{darkblue}{rgb}{0,0,0.545}

\frenchspacing


% ------------------------------------------------------------ %

\title{Finding the missing predictor}
\date{\today} %original date of receipt (as indicated by the Editor)
%
% ------------------------------------------------------------ %

\begin{document}
	\maketitle
	
\textbf{Storyline}:
	\begin{enumerate}
		\item In spatial analyses of species occurrence, residual spatial autocorrelation (rSAC) is a frequent problem. 
		Sources of rSAC are unknown and can be of varying nature: dispersal, missing predictor, spatially varying observation error, ...
		\item Frequently applied techniques to deal with rSAC are parametric models with a spatially parameterised variance-covariance matrix (see SAC-update). Less popular but appealing are methods that estimate latent predictors. These methods attempt to build a (set of) predictor(s) that represent unobserved, spatially autocorrelated, covariates which cause the observed rSAC. While these estimated surfaces might contain information about unobserved or disregarded data-generating processes (ecological or observation process), they are typically not further considered in the analysis. 
		\item Problem: we don't know how well and in which situations the developed methods can produce surfaces that resemble missing predictors. 
		\item Here we test several situations in an SDM-like analysis to quantify the potential of latent predictor estimating to identify missing predictors. For this we use mostly simulated data, but demonstrate the approach also in a case study modelling species richness.

	\end{enumerate}	

\end{document}