\documentclass[11pt]{article}
%\pdfoutput=1
% ------------------------------------------------------------ %

\usepackage{amsmath}

\usepackage{authblk} % for affiliations
\renewcommand\Affilfont{\itshape\footnotesize}
\setlength{\affilsep}{1em}

\usepackage[margin=10pt, font=small, labelfont=bf, tableposition=top]{caption}  % adjust caption properties

\usepackage[margin=1in]{geometry} % document margin
\usepackage{graphbox}
\usepackage{graphicx} 

\usepackage{hyperref}
\hypersetup{colorlinks=true, breaklinks=true, linkcolor=darkblue, menucolor=darkblue, urlcolor=darkblue, citecolor=darkblue}

%\usepackage{libertine}

\usepackage{mathtools} 
\DeclarePairedDelimiter{\ceil}{\lceil}{\rceil}

% enhanced tables
\usepackage{multicol}
\usepackage{multirow}
\usepackage{booktabs}
\usepackage{tabularx}
\newcolumntype{L}[1]{>{\raggedright\arraybackslash}p{#1}} %
\newcolumntype{C}[1]{>{\centering\arraybackslash}p{#1}} % 
\newcolumntype{R}[1]{>{\raggedleft\arraybackslash}p{#1}} % 

\usepackage{natbib} % bibliography
\bibpunct{(}{)}{,}{a}{}{,}  % adjust punctuation in references



\usepackage{tikz}

\usepackage{xcolor}
\definecolor{darkblue}{rgb}{0,0,0.545}

\frenchspacing


% ------------------------------------------------------------ %
\author[1]{Carsten F. Dormann}
\author[1]{Severin Hauenstein}
\author[2]{Jussi M\"akinen}
\affil[1]{Department of Biometry and Environmental System Analysis, University of Freiburg, 79106 Freiburg, Germany}
\affil[2]{Organismal and Evolutionary Biology Research Program, Faculty of Biological and Environmental Sciences, University of Helsinki, Helsinki, Finland}

\title{Finding the missing predictor}
\date{\today} %original date of receipt (as indicated by the Editor)
%
% ------------------------------------------------------------ %

\begin{document}
	\maketitle
	
\section{Storyline}
	\begin{enumerate}
		\item In spatial analyses of species occurrence, residual spatial autocorrelation (rSAC) is a frequent problem. 
		Sources of rSAC are unknown and can be of varying nature: dispersal, missing predictor, spatially varying observation error \citep{Dormann2007}.
		\item Frequently applied techniques to deal with rSAC are parametric models with a spatially parametrised variance-covariance matrix (see SAC-update). Less popular but appealing are methods that estimate latent predictors. These methods attempt to build a (set of) predictor(s) that represent unobserved, spatially autocorrelated, covariates which cause the observed rSAC. While these estimated surfaces might contain information about unobserved or disregarded data-generating processes (ecological or observation process), they are typically not further considered in the analysis. 
		\item Problem: we don't know how well and in which situations the developed methods can produce surfaces that resemble missing predictors. 
		\item Here we test several situations in an SDM-like analysis to quantify the potential of latent predictor estimating to identify missing predictors. For this we use mostly simulated data, but demonstrate the approach also in a case study modelling species richness.
	\end{enumerate}	

\section{Objectives}
\begin{itemize}
	\item Identify missing predictors: does the latent predictor surface provide information about misspecified models?
	\item Are some methods (Gaussian Process Models, Wavelets, Spatial Eigenvector Mapping, Generalised Additive Models) better than others for this purpose?
	\item If it works in principle, where can things go wrong: different causes of rSAC, biased estimates?
\end{itemize}

\section{Study design}
\begin{enumerate}
	\item simulate data using FReibier::simData() with dataset **2 (omitted predictor as cause for rSAC)
	\item fit Gaussian Process Models, Wavelets, Spatial Eigenvector Mapping, Generalised Additive Models using the 'misspecified' model structure
	\item check rSAC
	\item predict latent predictor $\hat{x}$
	\item Output:
	\begin{itemize}
		\item Median absolute difference (MAD) between left-out covariate $x$ and estimated latent predictor $\hat{x}$
		\item Bias on $\beta$-estimates (Can we use RMSE?)
	\end{itemize}
\end{enumerate}

\subsection{Scenarios}
\begin{itemize}
	\item 2 distributions: Gaussian, Bernoulli
	\item Gradient of $\beta$-values in simulating the 'missing' (i.e.~unknown predictor)
	\item Multiple (start with two) missing predictors
	\item Missing predictor as part of an interaction term (?)
	\item Dispersal as additional source of rSAC (couple with CAR-like approach?)
\end{itemize}

\subsection{Case study}
Repeat the study on species richness by \citet{Mahecha2008} using the latent-predictor methods (see above) with the aim to identify the variation in survey effort as a missing predictor. 

% References
\bibliographystyle{apalike} 
\bibliography{../literature/mimiX}

\end{document}